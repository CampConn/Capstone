\documentclass[10pt,journal,compsoc, draftclsnofoot,onecolumn]{IEEEtran}

\usepackage{graphicx}
\usepackage{subcaption}
\usepackage{epstopdf}
\usepackage{amssymb}                                         
\usepackage{amsmath}                                         
\usepackage{amsthm}                                          

\usepackage{alltt}                                           
\usepackage{float}
\usepackage{color}
\usepackage{url}


\usepackage{balance}
\usepackage{enumitem}
\usepackage{pstricks, pst-node}

\usepackage[margin=0.75in]{geometry}
\geometry{textheight=8.5in, textwidth=6in}

\newcommand{\cred}[1]{{\color{red}#1}}
\newcommand{\cblue}[1]{{\color{blue}#1}}
\usepackage{geometry}

\usepackage{hyperref}
\usepackage{fancyvrb}
\usepackage{color}
\usepackage[latin1]{inputenc}


\makeatletter
\def\PY@reset{\let\PY@it=\relax \let\PY@bf=\relax%
    \let\PY@ul=\relax \let\PY@tc=\relax%
    \let\PY@bc=\relax \let\PY@ff=\relax}
\def\PY@tok#1{\csname PY@tok@#1\endcsname}
\def\PY@toks#1+{\ifx\relax#1\empty\else%
    \PY@tok{#1}\expandafter\PY@toks\fi}
\def\PY@do#1{\PY@bc{\PY@tc{\PY@ul{%
    \PY@it{\PY@bf{\PY@ff{#1}}}}}}}
\def\PY#1#2{\PY@reset\PY@toks#1+\relax+\PY@do{#2}}

\expandafter\def\csname PY@tok@gd\endcsname{\def\PY@tc##1{\textcolor[rgb]{0.63,0.00,0.00}{##1}}}
\expandafter\def\csname PY@tok@gu\endcsname{\let\PY@bf=\textbf\def\PY@tc##1{\textcolor[rgb]{0.50,0.00,0.50}{##1}}}
\expandafter\def\csname PY@tok@gt\endcsname{\def\PY@tc##1{\textcolor[rgb]{0.00,0.25,0.82}{##1}}}
\expandafter\def\csname PY@tok@gs\endcsname{\let\PY@bf=\textbf}
\expandafter\def\csname PY@tok@gr\endcsname{\def\PY@tc##1{\textcolor[rgb]{1.00,0.00,0.00}{##1}}}
\expandafter\def\csname PY@tok@cm\endcsname{\let\PY@it=\textit\def\PY@tc##1{\textcolor[rgb]{0.25,0.50,0.50}{##1}}}
\expandafter\def\csname PY@tok@vg\endcsname{\def\PY@tc##1{\textcolor[rgb]{0.10,0.09,0.49}{##1}}}
\expandafter\def\csname PY@tok@m\endcsname{\def\PY@tc##1{\textcolor[rgb]{0.40,0.40,0.40}{##1}}}
\expandafter\def\csname PY@tok@mh\endcsname{\def\PY@tc##1{\textcolor[rgb]{0.40,0.40,0.40}{##1}}}
\expandafter\def\csname PY@tok@go\endcsname{\def\PY@tc##1{\textcolor[rgb]{0.50,0.50,0.50}{##1}}}
\expandafter\def\csname PY@tok@ge\endcsname{\let\PY@it=\textit}
\expandafter\def\csname PY@tok@vc\endcsname{\def\PY@tc##1{\textcolor[rgb]{0.10,0.09,0.49}{##1}}}
\expandafter\def\csname PY@tok@il\endcsname{\def\PY@tc##1{\textcolor[rgb]{0.40,0.40,0.40}{##1}}}
\expandafter\def\csname PY@tok@cs\endcsname{\let\PY@it=\textit\def\PY@tc##1{\textcolor[rgb]{0.25,0.50,0.50}{##1}}}
\expandafter\def\csname PY@tok@cp\endcsname{\def\PY@tc##1{\textcolor[rgb]{0.74,0.48,0.00}{##1}}}
\expandafter\def\csname PY@tok@gi\endcsname{\def\PY@tc##1{\textcolor[rgb]{0.00,0.63,0.00}{##1}}}
\expandafter\def\csname PY@tok@gh\endcsname{\let\PY@bf=\textbf\def\PY@tc##1{\textcolor[rgb]{0.00,0.00,0.50}{##1}}}
\expandafter\def\csname PY@tok@ni\endcsname{\let\PY@bf=\textbf\def\PY@tc##1{\textcolor[rgb]{0.60,0.60,0.60}{##1}}}
\expandafter\def\csname PY@tok@nl\endcsname{\def\PY@tc##1{\textcolor[rgb]{0.63,0.63,0.00}{##1}}}
\expandafter\def\csname PY@tok@nn\endcsname{\let\PY@bf=\textbf\def\PY@tc##1{\textcolor[rgb]{0.00,0.00,1.00}{##1}}}
\expandafter\def\csname PY@tok@no\endcsname{\def\PY@tc##1{\textcolor[rgb]{0.53,0.00,0.00}{##1}}}
\expandafter\def\csname PY@tok@na\endcsname{\def\PY@tc##1{\textcolor[rgb]{0.49,0.56,0.16}{##1}}}
\expandafter\def\csname PY@tok@nb\endcsname{\def\PY@tc##1{\textcolor[rgb]{0.00,0.50,0.00}{##1}}}
\expandafter\def\csname PY@tok@nc\endcsname{\let\PY@bf=\textbf\def\PY@tc##1{\textcolor[rgb]{0.00,0.00,1.00}{##1}}}
\expandafter\def\csname PY@tok@nd\endcsname{\def\PY@tc##1{\textcolor[rgb]{0.67,0.13,1.00}{##1}}}
\expandafter\def\csname PY@tok@ne\endcsname{\let\PY@bf=\textbf\def\PY@tc##1{\textcolor[rgb]{0.82,0.25,0.23}{##1}}}
\expandafter\def\csname PY@tok@nf\endcsname{\def\PY@tc##1{\textcolor[rgb]{0.00,0.00,1.00}{##1}}}
\expandafter\def\csname PY@tok@si\endcsname{\let\PY@bf=\textbf\def\PY@tc##1{\textcolor[rgb]{0.73,0.40,0.53}{##1}}}
\expandafter\def\csname PY@tok@s2\endcsname{\def\PY@tc##1{\textcolor[rgb]{0.73,0.13,0.13}{##1}}}
\expandafter\def\csname PY@tok@vi\endcsname{\def\PY@tc##1{\textcolor[rgb]{0.10,0.09,0.49}{##1}}}
\expandafter\def\csname PY@tok@nt\endcsname{\let\PY@bf=\textbf\def\PY@tc##1{\textcolor[rgb]{0.00,0.50,0.00}{##1}}}
\expandafter\def\csname PY@tok@nv\endcsname{\def\PY@tc##1{\textcolor[rgb]{0.10,0.09,0.49}{##1}}}
\expandafter\def\csname PY@tok@s1\endcsname{\def\PY@tc##1{\textcolor[rgb]{0.73,0.13,0.13}{##1}}}
\expandafter\def\csname PY@tok@sh\endcsname{\def\PY@tc##1{\textcolor[rgb]{0.73,0.13,0.13}{##1}}}
\expandafter\def\csname PY@tok@sc\endcsname{\def\PY@tc##1{\textcolor[rgb]{0.73,0.13,0.13}{##1}}}
\expandafter\def\csname PY@tok@sx\endcsname{\def\PY@tc##1{\textcolor[rgb]{0.00,0.50,0.00}{##1}}}
\expandafter\def\csname PY@tok@bp\endcsname{\def\PY@tc##1{\textcolor[rgb]{0.00,0.50,0.00}{##1}}}
\expandafter\def\csname PY@tok@c1\endcsname{\let\PY@it=\textit\def\PY@tc##1{\textcolor[rgb]{0.25,0.50,0.50}{##1}}}
\expandafter\def\csname PY@tok@kc\endcsname{\let\PY@bf=\textbf\def\PY@tc##1{\textcolor[rgb]{0.00,0.50,0.00}{##1}}}
\expandafter\def\csname PY@tok@c\endcsname{\let\PY@it=\textit\def\PY@tc##1{\textcolor[rgb]{0.25,0.50,0.50}{##1}}}
\expandafter\def\csname PY@tok@mf\endcsname{\def\PY@tc##1{\textcolor[rgb]{0.40,0.40,0.40}{##1}}}
\expandafter\def\csname PY@tok@err\endcsname{\def\PY@bc##1{\setlength{\fboxsep}{0pt}\fcolorbox[rgb]{1.00,0.00,0.00}{1,1,1}{\strut ##1}}}
\expandafter\def\csname PY@tok@kd\endcsname{\let\PY@bf=\textbf\def\PY@tc##1{\textcolor[rgb]{0.00,0.50,0.00}{##1}}}
\expandafter\def\csname PY@tok@ss\endcsname{\def\PY@tc##1{\textcolor[rgb]{0.10,0.09,0.49}{##1}}}
\expandafter\def\csname PY@tok@sr\endcsname{\def\PY@tc##1{\textcolor[rgb]{0.73,0.40,0.53}{##1}}}
\expandafter\def\csname PY@tok@mo\endcsname{\def\PY@tc##1{\textcolor[rgb]{0.40,0.40,0.40}{##1}}}
\expandafter\def\csname PY@tok@kn\endcsname{\let\PY@bf=\textbf\def\PY@tc##1{\textcolor[rgb]{0.00,0.50,0.00}{##1}}}
\expandafter\def\csname PY@tok@mi\endcsname{\def\PY@tc##1{\textcolor[rgb]{0.40,0.40,0.40}{##1}}}
\expandafter\def\csname PY@tok@gp\endcsname{\let\PY@bf=\textbf\def\PY@tc##1{\textcolor[rgb]{0.00,0.00,0.50}{##1}}}
\expandafter\def\csname PY@tok@o\endcsname{\def\PY@tc##1{\textcolor[rgb]{0.40,0.40,0.40}{##1}}}
\expandafter\def\csname PY@tok@kr\endcsname{\let\PY@bf=\textbf\def\PY@tc##1{\textcolor[rgb]{0.00,0.50,0.00}{##1}}}
\expandafter\def\csname PY@tok@s\endcsname{\def\PY@tc##1{\textcolor[rgb]{0.73,0.13,0.13}{##1}}}
\expandafter\def\csname PY@tok@kp\endcsname{\def\PY@tc##1{\textcolor[rgb]{0.00,0.50,0.00}{##1}}}
\expandafter\def\csname PY@tok@w\endcsname{\def\PY@tc##1{\textcolor[rgb]{0.73,0.73,0.73}{##1}}}
\expandafter\def\csname PY@tok@kt\endcsname{\def\PY@tc##1{\textcolor[rgb]{0.69,0.00,0.25}{##1}}}
\expandafter\def\csname PY@tok@ow\endcsname{\let\PY@bf=\textbf\def\PY@tc##1{\textcolor[rgb]{0.67,0.13,1.00}{##1}}}
\expandafter\def\csname PY@tok@sb\endcsname{\def\PY@tc##1{\textcolor[rgb]{0.73,0.13,0.13}{##1}}}
\expandafter\def\csname PY@tok@k\endcsname{\let\PY@bf=\textbf\def\PY@tc##1{\textcolor[rgb]{0.00,0.50,0.00}{##1}}}
\expandafter\def\csname PY@tok@se\endcsname{\let\PY@bf=\textbf\def\PY@tc##1{\textcolor[rgb]{0.73,0.40,0.13}{##1}}}
\expandafter\def\csname PY@tok@sd\endcsname{\let\PY@it=\textit\def\PY@tc##1{\textcolor[rgb]{0.73,0.13,0.13}{##1}}}

\def\PYZbs{\char`\\}
\def\PYZus{\char`\_}
\def\PYZob{\char`\{}
\def\PYZcb{\char`\}}
\def\PYZca{\char`\^}
\def\PYZam{\char`\&}
\def\PYZlt{\char`\<}
\def\PYZgt{\char`\>}
\def\PYZsh{\char`\#}
\def\PYZpc{\char`\%}
\def\PYZdl{\char`\$}
\def\PYZti{\char`\~}
% for compatibility with earlier versions
\def\PYZat{@}
\def\PYZlb{[}
\def\PYZrb{]}
\makeatother


\begin{document}

\title{
Progress Report\\
3D Object Pose Tracking for Robotics Grasping \\
CS461 Fall 2018
}
\author{Connor Campbell, Chase McWhirt and Jiawei Mo}

\maketitle

\begin{abstract}
The purpose of the project is to find the best of three methods for masking objects in two video feeds in real time.
Progress has mostly been researching, planning, and documenting what will go into a project like this.
There's a few obstacles that will have to be overcome, including the semantics of implementation, communication, and meeting the time line.
There is no code currently, but the client will have a data set ready by the end of Fall term.
Finally, a list of weekly summaries is given.
\end{abstract}

\IEEEdisplaynontitleabstractindextext
\IEEEpeerreviewmaketitle

\newpage
\pagebreak
\tableofcontents
\pagebreak

\section{Purpose}

The purpose of this project is to find the best method of creating a mask over a desired object in real time.
The desired object will be a robotic arm.
As stretch goals, the best masking method will also be used to define the space in which the object to be manipulated by the robotic arm exists in.
A final stretch goal would be to ensure both methods can run simultaneously without compatibility issues.


\section{Goals}

There are eight primary goals, as well as two stretch goals.

\begin{enumerate}
\item Defining masks over the data set using Gimp or Photoshop.
\item Implementing k-means clustering over all data. 
This will become the straw man to compare other implementations against.
\item Organize the data into groups by using a nearest neighbor algorithm.
\item Implement k-means clustering for each group defined in 3.
\item Implement a neural network for each group defined in 3.
\item Implement a support vector machine for each group defined in 3.
\item Test implementations over 10\% of data.
This data can not be used for training.
\item Report results and define best implementation from 4, 5, and 6.
\end{enumerate}

\noindent
There are two stretch goals.

\begin{enumerate}
\item Implement best method found in 8 to define the object to be manipulated by the arm.
\item Ensure both methods can run effectively simultaneously.
\end{enumerate}


\section{Progress}

There has been a lot of research collected, planning, and documentation written about this project.
Primarily, the research has been oriented around understanding the goals established by the client.
There is also some research about other tools that will be useful during implementation from technical reviews.
A time line has been created to help guide the project to completion.
Documentation has been written about the problem in general, the specific requirements that need to solved, the design of the general solution, and this progress report.
There has also been individual based documentation, allowing for more opportunity to research and plan, such as in the technical review.


\section{Obstacles and Solutions}

\subsection{Communication}
\noindent
Communication has been challenging overall.
At the start of the term, email was used primarily for communication.
With lack of consistently checking email, communication could take much longer than necessary.
This obstacle has been mostly resolved now that group 15 uses Slack to communicate.


\subsection{Following the Time Line}
\noindent
This could be extremely challenging due to there being three implementations.
The current time line is functional, but seems very tight, and not friendly to unforeseen time constraints presented by other classes.
Even if the time line is followed as expected, it could become way more stressful than anticipated.
I think the best solution would be to change one of the planned implementations into a stretch goal.
This could either be the support vector machine implementation or the custom neural network implementation.


\subsection{Read Images' Data}
\noindent
The problem is that the neural networking process can not recognize the input images.
Thus, it is necessary to transfer the images to digits and then feed data to neural networking process.
The solution is to store images and information about them in matrices that the software can understand, as well as the ability to manipulate that information.

\noindent 
In order to store and manipulate images, OpenCV will be used.
OpenCV is an open source C/C++ library aimed at solving computer vision problems \cite{1:online}.
OpenCV stores the pixels of an image in a matrix.
Depending on the type of picture, each pixel can hold various values representing their color.
Each value in the matrix can represent a variable of RGB (BGR instead of RGB in OpenCV functions) or gray scale.
A pixel's color is described and manipulated through the use of a channel, which stores many values related to the pixel and automatically adjusts related values when one is changed.
For instance, it can convert from BGR to HSV and vice versa, which allows for HSV to be used even if the image is based on the RGB color system.
For the purposes of this project, OpenCV will convert the colors of the pixels in an image into a set of values that the software will understand.


\subsection{Increase Color Contrast}
Basically, this process is called Masking.
This solution is considered as a result of improving pixels determination.
Each image will have three status.
They are object, robotics hand and background.
Every single pixel is holding one of them but it might be hard to determine pixel's status at the outline, which is the boundary of two different things.
To make an assumption, there is an red object and the background is white.
Pixels could be light red at the boundary between object and the background.
A mask process will assign the pixel to either object or background to increase the color difference.
Essentially, the mask operation looks for sudden changes in color within an image to determine approximate boundaries for different objects.
After obtaining an approximate area, the neural network will be tasked with providing a more precise identification.
The client suggests using the K-means algorithm for this as well.


\section{Code Insertion}
So far there has not been any actual coding, as the early stage of the project is simply to gather information and build a foundation of knowledge that will avoid problems impeding progress.
See the presentation of the idea behind the algorithm and technologies. 


%\newpage
%\section{Weekly Summary}
% This section is going to be the table
%yeah, that's what I was planning
%**combine yours

%\subsection{Week One}
%The project information had not been posted and team had not been assigned.
%The first week was a buffer time to get back to academic year and be adapted to courses. 

%\subsection{Week Two}
%No progress this week.

%\subsection{Week Three}
%Applicants filed for their desired projects.
%Groups were designated and the Github repository of the group is established for the future progress tracing. 

%\subsection{Week Four}
%Group met with client to build up a perspective of the project and the expectation of the final version.
%A stationary schedule for communicating was not confirmed.
%Group was starting the Problem Statement assignment.

%\subsection{Week Five}
%Group met with client in order to understand detailed requirements of the project.
%Group came out the outline for Requirements Document assignment.
%Moreover, Gantt Chart was generated and included into the document to afford the group a solid workflow in the future.
%Knowing that the robotic grasping used OpenRAVE to capture the scene through cameras, Jiawei was looking at OpenRAVE examples online and some researches of this open source. 

%\subsection{Week Six}
%Group decided the Team Standards that depicts common regulations of the team.
%Also, group kept working on Requirements Document.
%The problem is that none of us in the group have machine learning experiences or taking the courses opened in spring term.
%The Requirements Document's machine learning part showed a weak and invalid implementation to TA.
%Group tried to revise it later after next meeting with the client.
%Further, Tech Review started. 

%\subsection{Week Seven}
%Week seven marked when the group worked on the Tech Review essay, a detailed document summarizing all the research and reading the group members had completed.
%Some challenges of this paper included understanding a specific problem in the whole project and doing researches to find other technologies that can replace the current solutions. 

%\subsection{Week Eight}
%Group met the client, which had narrowed down the project's goals and design.
%Preparing the Design Document and asking the client the machine learning part with detailed questions and the anticipated outcomes to confirm the solution is valid.
%Jiawei was looking at OpenCV function library that could be implemented into the project in order to transfer images to digits. 

%\subsection{Week Nine}
%Week nine consisted of teamwork on researching and Design Document with feedback gave by the client.
%At this time, group had a stronger understanding of the machine learning section and how to implement it efficiently, which includes data feeding, data processing, training and outcome comparison. 

%\subsection{Week Ten}
%Group completed the Design Document and prepared the Progress Update assignment including a video/presentation.
%A Todo list was generated and shared on the Google doc.
%Each member was assigned a few fragments of the project to do presentation and videos would be combined at the end.
%Also, a Github link that contains essays was sent to the client for a verification. 


\section{Retrospective Table}
\begin{center}
 \begin{tabular}{|p{0.2\linewidth}|p{0.2\linewidth}|p{0.2\linewidth}|p{0.2\linewidth}|}
  \hline
 Week & Positives & Deltas & Actions \\ [0.5ex]
 \hline\hline
 

4
&
Group met with client to build up a perspective of the project and the expectation of the final version.
Group was starting the Problem Statement assignment.
&
A stationary schedule for communicating was not confirmed.
&

\\ \hline

5
&
Group met with client in order to understand detailed requirements of the project.
Group came out the outline for Requirements Document assignment.
Moreover, Gantt Chart was generated and included into the document to afford the group a solid workflow in the future.
Knowing that the robotic grasping used OpenRAVE to capture the scene through cameras, Jiawei was looking at OpenRAVE examples online and some researches of this open source. 
&

&

\\ \hline
 \end{tabular}
\end{center}

\section{Retrospective Table}
\begin{center}
 \begin{tabular}{|p{0.2\linewidth}|p{0.2\linewidth}|p{0.2\linewidth}|p{0.2\linewidth}|}
  \hline
 Week & Positives & Deltas & Actions \\ [0.5ex]
 \hline\hline

6
&
Group decided the Team Standards that depicts common regulations of the team.
Also, group kept working on Requirements Document.
Further, Tech Review started. 
&
The problem is that none of us in the group have machine learning experiences or taking the courses opened in spring term.
The Requirements Document's machine learning part showed a weak and invalid implementation to TA.
&
Group attempted to revise Requirements Document later after next meeting with the client.
\\ \hline

7
&
Group continued working on Tech Review
&
We decided that we need to improve communication between group members
&
We set up a channel on Slack for more immediate communication than email
\\ \hline


8
&
Group met the client, which had narrowed down the project's goals and design.
Preparing the Design Document and asking the client the machine learning part with detailed questions and the anticipated outcomes to confirm the solution is valid.
Jiawei was looking at OpenCV function library that could be implemented into the project in order to transfer images to digits. 
&

&

\\ \hline

 \end{tabular}
\end{center}

\section{Retrospective Table}
\begin{center}
 \begin{tabular}{|p{0.2\linewidth}|p{0.2\linewidth}|p{0.2\linewidth}|p{0.2\linewidth}|}
  \hline
 Week & Positives & Deltas & Actions \\ [0.5ex]
 \hline\hline

9
&
Week nine consisted of teamwork on researching and Design Document with feedback gave by the client.
At this time, group had a stronger understanding of the machine learning section and how to implement it efficiently, which includes data feeding, data processing, training and outcome comparison. 
&

&

\\ \hline

10
&
Group completed the Design Document and prepared the Progress Update assignment including a video/presentation.
A Todo list was generated and shared on the Google doc.
Each member was assigned a few fragments of the project to do presentation and videos would be combined at the end.
Also, a Github link that contains essays was sent to the client for a verification. 
&

&

\\ \hline

 
 
 
 
 
 \end{tabular}
\end{center}


\newpage
% bibliography
\nocite{*}%if nothing is referenced it will still show up in refs
\bibliographystyle{ieeetr}
\bibliography{refs}
%end bibliography
\end{document}